% LaTeX layout by Jonas Kahler, jonas@derkahler.de
% HashTux SAD Document
% Group Tux:
% Aman Dirar, Jerker Ersare, Jonas Kahler, Dennis Karlberg
% Niklas le Comte, Marco Trifance, Ivo Vryashkov
% Chapter 8 - Data View
\chapter{Data View}
This chapter gives a summary of what is stored in databases.

%% Social Media Post Document Format
\section{Social Media Post Document Format}
Stored in the \textit{``hashtux''} database. \\ \\
{\tabulinesep=1.4mm
\begin{tabu}{|l|l|}
\hline
\taburowcolors{gray!25..white}
Field & Purpose \\
\hline
\taburowcolors{}
search\textunderscore term & \parbox[t]{88mm}{The search term.} \\
\hline
service & \parbox[t]{88mm}{Social media service where it was posted i.e.
   Twitter, Instagram or YouTube.} \\
\hline
insert\textunderscore timestamp & \parbox[t]{88mm}{The time the internal data
   object was created.} \\
\hline
service\textunderscore id & \parbox[t]{88mm}{External item ID, as retrieved
   from social media service.} \\
\hline
timestamp & \parbox[t]{88mm}{The time the feed was posted on the social media
   service.} \\
\hline
text & \parbox[t]{88mm}{The text content of the post.} \\
\hline
language & \parbox[t]{88mm}{The language of the post (omitted by Instagram, not
   supported).} \\
\hline
view\textunderscore count & \parbox[t]{88mm}{The view count of the post on the
   social media (not applicable for Instagram).} \\
\hline
likes & \parbox[t]{88mm}{Amount of likes retrieved from social media.} \\
\hline
tags & \parbox[t]{88mm}{A list of tags related to the post.} \\
\hline
resource\textunderscore link\textunderscore high & \parbox[t]{88mm}{URL to
   high-definition resource (Instagram provides two image resolutions. Where not
   applicable, put the available resource link in both fields).} \\
\hline
resource\textunderscore link\textunderscore low & \parbox[t]{88mm}{URL to
   low-definition resource (Instagram provides two image resolutions. Where not
   applicable, put the available resource link in both fields).} \\
\hline
content\textunderscore type & \parbox[t]{88mm}{The content type of the post
   i.e. text, image or video.} \\
\hline
free\textunderscore text\textunderscore name & \parbox[t]{88mm}{Display name if
   it is separate to the username on that service (currently only applicable to
   Twitter).} \\
\hline
username & \parbox[t]{88mm}{The username of the poster.} \\
\hline
user\textunderscore id & \parbox[t]{88mm}{User ID as retrieved from social
   media.} \\
\hline
\end{tabu}}
%%%%% Page break %%%%%
{\tabulinesep=1.4mm
\begin{tabu}{|l|l|}
\hline
\taburowcolors{gray!25..white}
Field & Purpose \\
\hline
\taburowcolors{}
\hline
profile\textunderscore link & \parbox[t]{90mm}{URL to the user's profile.} \\
\hline
profile\textunderscore image\textunderscore url & \parbox[t]{90mm}{URL to the
   user's profile picture (only applicable to Twitter).} \\
\hline
date\textunderscore string & \parbox[t]{90mm}{Formatted date of when the post
   was made (only applicable to Twitter).} \\
\hline
location & \parbox[t]{90mm}{Data is social media service specific and can vary
but in general is a tuple of the form \{latitude, longitude\}. Can include
additional information if provided by the social media service.} \\
\hline
\end{tabu}}

%% User Habit Data Document Format
\section{User Habit Data Document Format}
Stored in the \textit{``hashtux\textunderscore userstats''} database. \\ \\
{\tabulinesep=1.4mm
\begin{tabu}{|l|l|}
\hline
\taburowcolors{gray!25..white}
Field & Purpose \\
\hline
\taburowcolors{}
\hline
term & \parbox[t]{92mm}{The search term the habit data belongs to.} \\
\hline
timestamp & \parbox[t]{92mm}{The time it was collected.} \\
\hline
session\textunderscore id & \parbox[t]{92mm}{The session id of the user.} \\
\hline
ip\textunderscore address & \parbox[t]{92mm}{The IP address of the user.} \\
\hline
browser & \parbox[t]{92mm}{The browser that was used.} \\
\hline
browser\textunderscore version & \parbox[t]{92mm}{The version of the used
   browser.} \\
\hline
platform & \parbox[t]{92mm}{The platform (OS) that was used.} \\
\hline
\end{tabu}} \\ \\ \\
In addition, all the options that were used for the request are also appended
when storing user habit data. They may include the following: \\ \\
{\tabulinesep=1.4mm
\begin{tabu}{|l|l|}
\hline
\taburowcolors{gray!25..white}
Field & Purpose \\
\hline
\taburowcolors{}
\hline
request\textunderscore type & \parbox[t]{97mm}{The type of request the user made
   i.e. search, update, heartbeat or stats. Required.} \\
\hline
services & \parbox[t]{97mm}{Any combination of services i.e. twitter, instagram
   or youtube (omitted if no services were filtered out).} \\
\hline
content\textunderscore type & \parbox[t]{97mm}{Any combination of content types
   i.e. image, video or text (omitted if no services were filtered out).} \\
\hline
language & \parbox[t]{97mm}{The language the request was filtered through
   (omitted if no language chosen was chosen).} \\
\hline
\end{tabu}}

%% Cached User Habit Data Document Format
\section{Cached User Habit Data Document Format}
Stored in the \textit{``hashtux\textunderscore userstats\textunderscore
cached\textunderscore data''} database. \newline
CouchDB is not very efficient when executing MapReduce functions on large amount
of data. To increase performance for the statistics page in the client UI, we
use the \textit{``hashtux\textunderscore userstats\textunderscore
cached\textunderscore data''} where we store documents consisting of map reduced
summarizations of the data in our \textit{``hashtux\textunderscore userstats''}
database. The documents contain \{key, value\} pairs where the key (string) is
e.g. search term, and value (int) is the amount of documents in the
\textit{``hashtux\textunderscore userstats''} database that have that key.
\newline
The documents have predefined names that consist of two parts, first the
dimension and then the interval e.g. ``search\textunderscore term\textunderscore
year''.  Each dimension name can be combined with any of the intervals.\newline
Available dimensions: ``search\textunderscore term'', ``browser'', ``platform'',
``platform\textunderscore browser'' (a combination of the fields platform and
browser), ``browser\textunderscore version''. \newline
Available intervals: ``today'' (last 24 hours), ``week'', ``month'', ``year''.